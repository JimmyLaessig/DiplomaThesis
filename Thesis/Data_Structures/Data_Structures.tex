\chapter{A functional out-of-core octree}
\section{Overview}
Modern point clouds are often to large too fit into memory, let alone video memory. In order to manage datasets whose size exceeds several gigabytes, the data must be stored in an structured way, such that data can be loaded in chunks efficiently. Depending on the camera's view and a level-of-detail ruling, only a subset of points is processed in the memory and displayed. This method is commonly know as out-of-core processing. One common way to introduce structure to point clouds is by storing the data in an octree. 


\section{Out-of-core octree}

An octree is a hierarchical datastructure in which each node represents a spatial region, defined by a 3d bounding box. If the number of elements in a node exceeds a threshold $n$, the node is partitioned into eight children, each representing one octant. If a node is partitioned, it's elements become a random subset of points of size $n$ from it's children, thus creating an efficient level-of-detail representation of the point cloud. 
Other implementations, such as Potree\cite{SCHUETZ-2016-POT}, where some points remain in the parent instead of being stored in a child node, are more efficient in terms of disc space. With our approach, each node can be viewed as self-contained, such that no points from predecessor nodes are needed to fully represent the point cloud for this region and level-of-detail.

Each node contains a set of points whose content is stored on the disc. The data is streamed from a file database into memory on access.


\section{Octree Operations}
	
\subsection{Map}
\subsection{Cull}
\subsection{Insert}


%In an functional program, changes in datastructures or variables are considered mutations and often introduce side-effects and 
%Changes in data structures Mutations of datastructures often introduce side-effects and race conditions. In a parallel environment, multiple threads execute %computations tasks that change the octree. 
