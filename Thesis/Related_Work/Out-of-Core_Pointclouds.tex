\section {Out-of-core Point-Clouds}
\label{sec:related_work_point_clouds}
As the size of modern point clouds often exceeds the available memory, special data structures and rendering techniques are needed that can handle such amounts of data. 

The QSplat System \cite{rusinkiewicz2000qsplat} was one of the earliest systems that was capable of handling datasets with well over hundred million points. It uses a hierarchy of bounding spheres that makes it easy to perform visibility culling and level-of-detail control. The hierarchy itself does not hold information of the containing points. Leaf nodes represent a single points sample from the original point cloud. A node is only rendered if it is a leaf or the decision is made that the benefit of recoursing to the children is too low, otherwise the children are traversed. If a node is rendered, a spherical splat is drawn with the size of the node. 

\par

Gobbetti and Marton \cite{gobbetti2004layered} propose a multi-resolution approach for rendering massive point clouds. The point cloud is stored in a binary tree. The root node of the level-of-detail tree contains a subset of uniformly distributed samples of the point cloud. The remaining points are distributed among the two subtree until the number of points deceeds a threshold value. Hence, no duplicate points are stored, making this technique very memory efficient. The structure handles view-frustum culling, occlusion culling, as well as view-dependent progressive transmission. This approach hides out-of-core latency by speculatively fetching data. 

\par

Wimmer and Scheiblauer \cite{wimmer2006instant} propose an algorithm to display enormous unprocessed point clouds at interactive rates without requiring costly postprocessing. The authors use a nested octree to handle out-of-core data and perform view-frustum culling, and view-dependent level-of-detail culling. Additionally, each node contains a memory optimized Sequential Point Tree (SPT). A SPT is a hierarchical point representation that allows rendering through a sequential processing on the GPU.  
\\
\\
Wand et al. \cite{wand2007interactive} describe an out-of-core multi-resolution data structure for real-time visualization and interactive editing of large point clouds. An octree is used as a dynamic multi-resolution data structure that handles out-of-core functionalities. Inner nodes of the octree contain downsampled point clouds with a sample spacing that is a fixed fraction of the node's bounding cube side length. The grid quantization allows for efficient dynamic updates to the octree, such as insertion and removal of points. 
\\
\\
Wenzel et al. \cite{wenzel2014out} propose an out-of-core octree that manages billions of points. It supports dynamic depth, defined by thresholds for the maximum data storage. The octree implements dynamic memory management, such that as much data is stored in memory, without exceeding the limits of the memory. It also supports tasks such as adding data, splitting and merging of nodes, and node history tracking. 
