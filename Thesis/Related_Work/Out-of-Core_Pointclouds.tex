\section {Out-of-Core Point-Clouds}
\label{sec:related_work_point_clouds}
As the size of modern point clouds often exceeds the available memory, solutions are found to handle such amounts of data. 
\\
Gobbetti and Marton \cite{gobbetti2004layered} propose a multi-resolution approach for rendering gigantic point clouds. The point cloud is stored at multiple resolutions using a binary tree that is split along its largest axis, with the leaves containing the highest resolution point cloud. Each partition contains a fixed number of points randomly sampled  from its children. Their approach hides out-of-core latency by speculatively fetching data. The structure handles view-frustum culling, occlusion culling, as well as view-dependent progressive transmission.
\\
\\
Wimmer and Scheiblauer \cite{wimmer2006instant} propose an algorithm to display enormous unprocessed point clouds at interactive rates without requiring long postprocessing. They use a nested octrees to handle out-of-core data, and perform view-frustum culling, and view-dependent level-of-detail. Additionally, each node contains a memory optimized Squential Point Tree (SPT). A SPT is a hierachichal point representation that allows rendering through a sequential processing on the GPU.  
\\
\\
Wand et al. \cite{wand2007interactive} describe an out-of-core multi-resolution data structure for real-time visualization and interactive editing of large point clouds. An octree is used as a dynamic multi-resolution data structure that handles out-o-core functionalities. Inner nodes of the octree contain downsampled point clouds with a sample spacing that is a fixed fraction of the node's bounding cube side length. The grid quantization allows for efficient dynamic updates to the octree, such as insertion and removal of points. 
\\
\\
Wenzel et al.  \cite{wenzel2014out} propose an out-of-core octree that manages billions of points. It supports dynamic depth, defined by thresholds for the maximum data storage. The octree implements dynamic memory management, such that as much data is stored in memory, without exceeding the limits of the memory. It also supports tasks such as adding data, splitting and merging of nodes, and node history tracking. 
