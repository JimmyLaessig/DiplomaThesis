\chapter{Introduction}
\newpage

\section{Motivation}

With the increasing availability of 3D sensors over the past years, structure finding in point clouds has received increasing interest as well. While different methods for generating point-cloud data exist, such as laser scanners, Microsoft Kinect or photogrammetric reconstructions, the presented data commonly lacks structure and semantic information.Documenting geomorphological erosion, monitoring urban and agricultural developments, mapping archeological sites and generating assets for entertainment industry are only some fields of application for point clouds. 
\\
Point-cloud datasets have grown in size at such a rapid rate that they are now simply too large to fit into system memory, let alone graphics card memory. Therefore, new solutions for out-of-core representations have emerged. In such a case, the point cloud is stored in a cached file on the hard drive and can therefore not be accessed directly. Based on a culling heuristic, chunks of point-cloud data are loaded into memory. 
This continuous swapping of data yields the disk speed as a potential bottleneck when it comes to performance. However, this exchange of data comes with the benefit that only chunks of data are in memory which is of interest to the user. 
\\
The objective of shape detection is to find homogeneous regions in point clouds with similar characteristics to help the user understand local and global structures. Moreover, it can be used to introduce semantic information into unstructured data, providing the user with more interaction possibilities. Current solutions, such as presented by Schnabel et al.\cite{schnabel-2007-efficient}\cite{schnabel-2007-ransac}, can already
produce precise segmentations of point clouds. The complexity of these algorithms increases with the size of the point cloud, making the computation for billions of points infeasible in real-time. However, when looking at raw numbers, the approach delivers promising results for point clouds of smaller size (<12.000 points) in a fraction of a second.
While the research fields of representation and rendering of point-clouds, as well as shape detection and segmentation,  have received tremendous interest lately, less research was focused on interactions and real-time processing. 

\\
This thesis proposes an alternative approach for multi-scale shape detection on local regions in point clouds and presents several interaction techniques that utilize semantic information in order to assist the user's workflow. 


\section{Problem Definition}


A popular way to introduce structure into point clouds is by using an octree to spatially divide the point cloud into processable chunks. 
Efficient storing and representation for out-of-core point clouds is discussed in numerous publications, Scheibelbauer\cite{scheiblauer-thesis}, Elseberg et. al\cite{elseberg2013one} or the Point Cloud Library\cite{rusu20113d}, to only name a few. However, as the use cases for this thesis are different, a custom solution to store point clouds with semantic information is needed. Additionally to storing points, the 





\section{Contributions}
\section{Structure of the Work}




