\chapter{Introduction}

\section{Motivation}

With the increasing availability of 3D sensors over the past years, structure finding in point clouds has received increasing interest as well. While different methods for generating point-cloud data exist, such as laser scanners, Microsoft Kinect or photogrammetric reconstructions, the presented data commonly lacks structure and semantic information.Documenting geomorphological erosion, monitoring urban and agricultural developments, mapping archeological sites and generating assets for entertainment industry are only some fields of application for point clouds. 
\\
\\
Point-cloud datasets have grown in size at such a rapid rate that they are now simply too large to fit into system memory, let alone graphics card memory. Therefore, new solutions for out-of-core representations have emerged. In such a case, the point cloud is stored in a cached file on the hard drive and can therefore not be accessed directly. Based on a culling heuristic, chunks of point-cloud data are loaded into memory. This introduces the benefit of only having chunks of data in memory that are of immediate interest to the user. However, the disk speed is a potential bottleneck when it comes to performance because of the continuous swapping of data.
\\
\\
The objective of shape detection is to find similar regions in point clouds with similar characteristics to help the user understand local and global structures. Moreover, it can be used to introduce semantic information into unstructured data, providing the user with more interaction possibilities. Current solutions, such as presented by Schnabel et al. \cite{schnabel-2007-efficient, schnabel-2007-ransac}, can already produce a precise segmentation of point clouds. The complexity of these algorithms increases with the size of the point cloud [// complexity always increases, but by how much (quadratic .. ) ], making the computation for billions of points infeasible in real-time. However, when looking at raw numbers, the approach delivers promising results for point clouds of smaller size (<12.000 points) in a fraction of a second.
While the research fields of representation and rendering of point-clouds, shape detection and segmentation, have received tremendous interest lately, less research focused on interactions and real-time processing. 
\\
This thesis proposes an alternative approach for multi-scale shape detection on local regions in point clouds and presents several interaction techniques that utilize semantic information to assist the user's workflow. This thesis is developed using the Aardvark platform, which is a functional-first incremental rendering engine written in F\#. This thesis showcases some functional aspects of computer graphics as well. 


\section{Problem Definition}

The size of point clouds, obtained by laser scanners or photogrammetric networks, often exceeds several million points. 
Efficient out-of-core solutions for point clouds are discussed in numerous publications, Scheibelbauer \cite{scheiblauer-thesis}, Elseberg et al. \cite{elseberg2013one} or the Point Cloud Library \cite{rusu20113d}, to only name a few. However, a custom solution that stores point clouds, enriched with semantic information, is needed.
\\
In scans of urban environments, many structures can be represented in a more memory-efficient way. Points that follow a wall can often be compressed to few triangles. Pillars often share similarities with cylinders. Detecting such shapes is an immense task that scales with the number of points and fails to be executed in real-time. When exploring a point cloud, immediate feedback of local geometry is useful, since it introduces additional information to the user. This task cannot be achieved without substantial preprocessing of the point cloud. 
\\
The task of precisely selecting regions or points of interest in point clouds can be tedious and cumbersome. By using 2D-interaction metaphors, it is challenging to the user to select spatially neighboring points that follow the same curvature (e.g., points on a wall), as the techniques do not know the desired depth boundaries of the selection region. Interactions across multiple views are needed to achieve such a selection.  Methods that use 3D information, such as a volumetric brush by Weyrich et al. \cite{weyrich2004post}, can make such tasks easier. By consulting the depth buffer each frame, the brush follows the curvature of the geometry. However, by reading pixels from the GPU, the rendering process is stalled. This technique reacts to occlusion such that the brush follows geometry in front of the wall. Thus, view changes are still required.


\section{Contributions}

The main contribution of this diploma thesis is the implementation of an alternative way of detecting shapes on multiple level-of-details in point clouds. Instead of performing shape detection on the complete point cloud at once, we use an approach that lets the user control the region in which shapes should be detected. By reducing those regions to a suitable size, a well-known shape detection algorithm can return meaningful results in interactive time, such that the user is presented with immediate feedback on his selection. 
\\
\\
Shapes are detected on multiple level-of-details in the point cloud. A clustering algorithm takes shapes from different \textit{level-of-detail}, based on a similarity heuristic, and creates a larger connected cluster of shapes. This cluster is used to present information on a larger scale to the user, rather than each shape separately. 
\\
\\ 
This thesis proposes several improvements to commonly known user interactions. \textit{Point Picking} and \textit{Region Selection} are improved by consulting the local geometry of the point cloud to assist the user. By using a shape as support, the interaction dimensions are reduced to the parameter space of the shapes, allowing to exclude unwanted points from interactions easily. 
\\
\\
Additionally,  a novel interaction technique is introduced that allows the user to increment the \textit{level-of-detail} locally along a shape. This helps the user to explore the structure of the point cloud in more detail. 


\section{Structure of the Work}

The structure of the work is as follows. Chapter \ref{chap:related_work} covers the related work for this thesis, including point-cloud rendering and out-of-core representations, shape detection, and segmentation and advanced user interactions on point clouds. Chapter \ref{chap:octree} describes the octree used for the out-of-core representation of the point cloud, as well as some metrics that further describe the content of an octree node. Chapter \ref{chap:shapeDetection} describes the algorithms used to detect primitive shapes in a point cloud and proposes a technique to cluster similar shapes into one coherent shape cluster used for user interactions. Chapter \ref{chap:systemDesign} discusses the application's features including the user-controlled shape detection and assisted interactions that utilize the detected shapes as support shape. Chapter \ref{chap:implementation} focuses on implementation details in a functional context. Results of the application are presented in Chapter \ref{chap:results}. Chapter \ref{chap:conclusion} concludes this thesis with a reflection on the application and an outlook on future work. 
