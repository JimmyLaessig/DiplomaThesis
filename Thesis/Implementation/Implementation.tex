\chapter{Implementation}


\section{Functional Programming}
\label{sec:funprog}
Functional programming is a paradigm that views every expression as a mathematical function. The result of each expression is either an elemental datatype or a functional type. A core difference to other programming paradigms is the equality of functions and data, such that function can be used as parameter for other functions. A function's type is defined from its parameters and result. A simple example of a function in \verb|F#|: 

\begin{lstlisting}[language=FSharp]
let square (i : float) : float = i * i 
\end{lstlisting}

This function is of type \verb|float -> float| and takes a \verb|float| as input and returns a \verb|float|. This functional type can be used as input parameter as well. A definition for such a function is the following example: 

\begin{lstlisting}[language=FSharp]
let compute (i: float)(f : float -> float) : float = f i
\end{lstlisting}
Its type is \verb|float -> (float -> float) -> float| and is used as such: 
\begin{lstlisting}[language=FSharp]
let result = compute 10.0 square
\end{lstlisting}

\verb|result| is an expression that executes the \verb|compute| function with arguments $10.0$ of type \verb|float| and \verb|square| of type \verb|float -> float|. Even tough this is just a simple example, it showcases the strength of function programming, such that that using functions as parameter allows the user to create complex and dynamic computations with ease. 
\\

Functional programming tries to mitigate mutations as much as possible. Mutations, e.g. change of a variable's state, introduces several risks to the program. While each expression in a function context is mathematically defined, such that parameter space and result space are fully defined, now values are returned that the program does not expect, thus mitigating undefined behavior. Moreover, mutation of an expression can change the behavior of expressions that depend in the mutated field, thus changing result, without changing the parameter. 
\\

\verb|F#|\cite{FSharp} is a function programming language developed and maintained by Microsoft\cite{Microsoft}. F\# is fully integrated in the .NET framework\cite{DotNet} and is build upon the Common Language Infrastructure\cite{CLI}, thus allowing it to use resources written in other languages, such as C\#\cite{CSharp}. Eventough F\# is a functional language, it also supports object-oriented programming, such that types with member functions can be used as well. 


\section{Aardvark}
The Aardvark platform\cite{aardvark} is a functional-first incremental rendering engine in active development at the VRVis Zentrum für Virtual Reality und Visualisierung\cite{vrvis}. It is an implementation of research results on the field of incremental rendering \cite{, lazy} and semantic shader composition\cite{haaser2014cosmo, haaser2014semantic}. 

The key feature of the Aardvark platform is incremental rendering. In a conventional engine, updates are performed periodically. each object in the scene is reevaluated, even tough the simulation or user-input did not yield any changes. Incremental rendering counters this overhead by reacting to changes, such that only parts that depend on changed values are reevaluated. This section describes some of Aardvark's key features that are used throughout this thesis. Section\ref{sec:adaptive} shows the language-specific constructs on how to build adaptive blocks that react to changes. A \textit{scene graph} represents an object hierarchy in the scene. Section \ref{sec:isg} describes the composition of a scene graph using only a handful of lines of code. 

 
\subsection{Adaptive - IMod - transact}
\label{sec:adaptive}
As already mentioned in Section \ref{sec:funprog} mutations are often the cause of undefined behavior and bugs. In order to introduce state-changeable variables into a functional environment, Aardvark provides the \verb|IMod<'a>| type. This generic type is a wrapper around a certain value, whose value might change over time. The platform contains an extensive implementation for a three-dimensional transformation, called \verb|Trafo3d|. In order to listen to changes for such a transformation, a type \verb|IMod<Trafo3d>| is needed. The following example shows the composition of a model transformation from a position, scale and rotation, all of which can might their value over time. 

\begin{lstlisting}[language = FSharp]

let position 	: IMod<V3d> = ... 
let scale 		: IMod<V3d> = ... 
let rotation	: IMod<V3d> = ...

let trafo : IMod<Trafo3d> = 
	adaptive {
	
		let! sc  = scale
		let! rot = rotation
		let! pos = position
		
		
		let S = Trafo3d.scale sc
		let R = Trafo3d.Rotation rot
		let T = Trafo3d.Translation pos
		
		return S * R * T
	}
\end{lstlisting}
The \verb|adaptive| block allows the system to keep track of the state of all IMods that are accessed using the \verb|let!| operator. The result of the adaptive block, again, is an IMod<Trafo3d>. If one of the accessed IMods changes its value, the code below, including the line of the \verb|let!| is reevaluated using the previously caches values. 
\\

To actively change a value, a subtype of \verb|IMod<'a>|, \verb|ModRef<'a>| is needed. The type \verb|ModRef<'a>| contains the functionality to allow changes to the value, thus causing reevaluation. All state transactions are collected and executed sequentially, thus reducing the number of reevaluations and circumventing race-conditions. In order to achieve that each change must be wrapped in a \verb|transact| type. 
The following example uses the Aardvark-specific mouse callback function to trigger a reevaluation based on mouse movement. The Move-callback is called each time the mouse moves and provides the user with the old position and the new position. The difference on the x-axis controls the y-value of the rotation. 

\begin{lstlisting}[language = FSharp]
let mouse : IMouse = ...

let rotation : ModRef<V3d> = Mod.init V3d.OOO

mouse.Move.Values.Add(fun (oldPos : PixelPosition, newPos : PixelPosition) -> 
	let delta = newPos.NormalizedPosition.X - oldPos.NormalizedPosition.X
	let angle = delta * 2.0 * Math.Pi
	
	let newRotation = rotation.GetValue() + V3d(0.0, angle, 0.0)
	
	transact(fun () -> Mod.change rotation newRotation)
	)
\end{lstlisting}

The \verb|rotation| can than be used within an \verb|adaptive| block like an \verb|IMod|. 
\\
Transactions are usually used to handle user input. However, asynchronous computations that produce results in parallel use transactions as well in order to mitigate race conditions and notify the system. 


\subsection{Scene graph composition}
\label{sec:isg}
A scenegraph (ISg) contains all information that are needed to render this object. A common paradigm in functional programming is to split functionality from data, such that the functions that utilize this data, are stored in a different namespace than the data. In combination with the pipe operator ($|>$) clean and easy-to-read code can be produced. The following example showcases the composition of an \verb|ISg|. 

\begin{lstlisting}[language = FSharp]

// Transformations
let trafo:	IMod<Trafo3d> = ... 
let view:		IMod<Trafo3d> = ...
let proj:  	IMod<Trafo3d> = ...

// IndexedGeometry contains triangles to render
let geometry : IndexedGeometry = ...

// The renderpass for this scenegraph
let renderPass : RenderPass = ...
// The shader (surface) to render the geometry with
let shader : IMod<ISurface> = ...


let sg = geometry	|> Sg.ofIndexedGeometry
					|> Sg.trafo trafo
					|> Sg.viewTrafo view
					|> Sg.projTrafo proj
					|> Sg.pass renderPass
					|> Sg.surface shader
\end{lstlisting}

The field \verb|sg| represents a scene graph with transformations, geometry, shader and render pass without the need of complicated constructors or setter functions. Furthermore an own implementation if \verb|ISg| can easily extend the functionality without reimplemented procedures for this specific type. 


\subsection{Functional Octree}
\section{System Design}
\section{Multi-Threaded Environment}
\section{Interaction Workflow}
\section{Benchmarks}