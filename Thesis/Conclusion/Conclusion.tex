\chapter{Conclusion and Future Work}
\label{chap:conclusion}


This chapter completes this thesis by presenting a final conclusion on the work and proposing future work.

\subsection{Out-of-core Representation of Point Clouds}

Modern point clouds often consist of several billion points and consume several gigabytes of memory. They are simply too large to fit into system memory as a whole. This thesis describes a functional octree structure that is able to handle the out-of-core functionality of the point cloud. Data is stored in chunks in a file and is loaded into system memory when needed. Additionally, the point cloud's octree nodes contain a subset of points from its children, thus creating a multi-scale representation of the point cloud that is used for rendering and interactions. 


\subsection{User-guided Shape Detection}

This thesis shows a alternative use of the shape detection algorithm by Schnabel et al.\cite{schnabel-2007-efficient} which lets the user control the regions that are segmented. By pointing to a region with the mouse, the system selects the best candidate node to be segmented. The number of point in this region is chosen in such a way that the shape detection delivers results in interactive time. This immediate feedback allows the user to use the detected shapes as support shapes for improved interactions. 


\subsection{Improvement of Interaction Techniques}

Several improvements to well-known two-dimensional interaction techniques are discussed in this thesis. We propose a way of prefiltering point for interactions, such that only points are considered that follow the curvature of a primitive shape that is picked by the user. Classic Point Picking is improved such that only points are picked that belong to the support shape. This is especially useful when trying to pick points that are otherwise occluded or lie on the edge of a structure. Multiple ways exist to select regions in a point cloud. We utilize a support shape to improve the Lasso Selection. Classic Lasso Selection selects all points, whose projection lies inside the lasso on the near plane. The user selects 'through' the point cloud, whereas when using a support shape, the user only selects points that lie on this support shape. The volumetric brush allows the user to select points that are in the foreground. Instead of consulting the depth buffer in order to retrieve the brush's position the position of the cursor on the support shape is used. Thus the brush does not follow the structures that are in the foreground but the curvature of the support shape. Therefore points can be selected that are occluded otherwise. 


\subsection{Novel Interaction to magnify Local Details}

The last task of this thesis was to design a novel interaction technique called \textit{Shape-Assisted Local Level-of-Detail Increment}. The \textit{level-of-detail} rendering of the point cloud only displays a set of points such that the graphics card can render the points in real time. Thus, features may get culled and are not rendered. The interaction technique collects point from nodes that are not rendered due to their \textit{level-of-detail}. These additional points follow the curvature of the support shape. Along the support shape, additional points are rendered that would otherwise be culled. This allows the user to get a more detailed look at structures. 


\section{Future Work}

As the focus of this thesis was the design of the user-guided shape detection and the assisted interactions, future work will focus on performance and robustness improvements on various fronts. The usage of a selection data structure, such as a Selection Octree\cite{scheiblauer2011out}, can improve the performance when selecting or editing the point cloud. Also, different ways can be explored to improve computation speed by taking advantage of the parallel architecture of the graphics card. 
\\

Shape detection is performed using an external library by Schnabel et al.\cite{schnabel-2007-software}. Multiple problems occur, such as non-termination or non-plausible shape matching. More work must be carried out in order to limit or eliminate these problems. The robustness of the shape detection, especially when detecting non-planar primitive shapes, heavily suffers from the weak constraints. Alternatively the shape detection can be implemented in such a way, that certain types of shapes are prioritized in order to reduce the amount of non-plausible shapes. 
\\

