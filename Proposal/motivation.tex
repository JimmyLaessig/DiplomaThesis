\section{Motivation \& Problem Definition}

With the increasing availability of 3D sensors over the past years, structure finding in point clouds has received increasing interest as well. While different methods for generating point-cloud data, such as laser scanners, Microsoft Kinect or photogrammetric reconstructions, exist, the presented data commonly lacks structure and semantic information. The objective of a shape-detection process is to find homogeneous regions in point clouds with similar characteristics to help the user understand local and global structures. Moreover, it can be used to introduce semantic information into unstructured data, providing the user with more interaction possibilities. Current solutions, such as presented by Schnabel et al. \cite{schnabel-2007-efficient}\cite{schnabel-2007-ransac}, can already precisely detect shapes and objects in point clouds. The complexity of these algorithms increases with the size of the point cloud, making the computation for billions of points infeasible in real-time. However, when looking at raw numbers, the approach by Schnabel et al. \cite{schnabel-2007-ransac} delivers promising results for point clouds of smaller size (\textless 12.000 points) in a fraction of a second. 
\\
Point-cloud datasets have grown in size at such a rapid rate that they are now simply too large to fit into system memory, let alone graphics card memory. Therefore, new solutions for out-of-core representations have emerged. In such a case, the point cloud is stored in a cached file on the hard drive and can therefore not be accessed directly. Based on a culling heuristic, chunks of point-cloud data are loaded into memory when being processed or rendered. This continuous swapping of data yields the disk speed as potential bottleneck when it comes to performance.
\\
This thesis presents an interactive approach to detect shapes on different scales using a \textbf{L}evel-\textbf{o}f-\textbf{D}etail (LoD) representation of a point cloud. Instead of segmenting the whole point cloud at once, the user's input is used to extract regions of interest that can be segmented within a fraction of a second, thus providing the user with immediate geometrical information on the currently focused region. Moreover, selecting points or regions in point clouds can be tedious and cumbersome. This thesis presents several interaction concepts that utilize the geometric information in order to streamline the user's workflow. 