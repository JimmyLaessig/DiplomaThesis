\documentclass[]{article}

\usepackage{amsmath}
\usepackage{amssymb}
\usepackage{graphicx}
\usepackage{epstopdf}
\usepackage{inputenc}
\usepackage{geometry}
\geometry{left=2.5cm,right=2.5cm,top=2.5cm,bottom=2.5cm}

%opening
\title{\textbf{Diploma Thesis Proposal}\\
	\Large Interactive Segmentation in Out-Of-Core Pointclouds for assisted User Interactions}

\author{Bernhard Rainer}



\begin{document}

\maketitle

\section{Motivation \& Problem Definition}

\section{Aim of the Work}

The target of this diploma thesis is to implement a system that allows the user to interact using geometric information that is extracted from the pointcloud in interactive time (> 0.5 seconds). The algorithm does not segment the whole pointcloud at once, but the current region of interest, indicated by the cursors' position and a focal distance. The pointcloud is representated by an octree datastructure that is capable of handling out-of-core memory updates for large pointclouds. The data is stored in the leave nodes of the octree, the inner nodes store a subset of points from its child nodes, thus creating an effective Level-of-Detail (LoD) representation. Using this LoD information and the users mouse input, a subset of nodes can be collected on which the segmentation algorithm can focus. This approach yields geometric an semantic information of the geometrical structure within interactive time. 

The segmentation algorithm is able to extract a minimal set of geometrical primitives, such as plane, spheres and cylinders, such that each point can be assigned to one of those primitives with a good-enough possibility. This step will follow the algorithm proposed by Schnabel et al. in \cite{schnabel-2007-efficient} and \cite{schnabel-2007-ransac}. 

% TODO
% From Primitive to polygon
% Collaple similar polygons
% Guided Picking

%Finding such primitives is one part of the segmentation, the second part is to collapse similar primitives together such that larger homogenous %structures are created. O-snap\cite{arikan-2013-osn} developed by  Arikan et al. at the VRVis proposes a guided approach on on snapping such polygons


\section{Methodology and Approach}

\section{Structure of Work}

\section{State-of-the-Art}

\section{Relevance to the Curriculum of Visual Computing}

This diploma thesis is focused on applications in the field of Visual Computing, in particular: 

\begin{itemize}
	\item 186.140 VU Echtzeitgraphik
	\item 186.166 VU Entwurf und Programmierung einer Rendering-Engine
	\item 104.319 VU Geometrie für Informatik
	\item 186.833 VU Visualisierung 2
	\item 186.191 VU Echtzeit-Visualisierung
\end{itemize}

\bibliographystyle{plain}
\bibliography{references}

\end{document}
