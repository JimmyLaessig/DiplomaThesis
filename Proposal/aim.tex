\section{Aim of the Work}
\label{sec:aim}

The target of this diploma thesis to implement a top-down shape detection algorithm that operates on a level-of-detail representation of a point-cloud, such that within a subset of the point-cloud shapes can be detected in near real-time. 
This approach extracts regions of interest, indicated by the users curor, current viewport a focal distance. The point cloud is represented as an octree data structure that is capable of handling out-of-core memory updates of the dataset. The points are stored in the leave nodes of the octree, inner nodes store a subset of points from its children, thus creating an efficient LoD representation. Using a LoD-heuristic paired with the users region of interest, a subset of nodes can be collected on which the segmentation algorithm should focus. 
\\
The segmentation algorithm is able to extract a minimal set of geometrical primitive shapes (plane, sphere, cylinder), such that with a certain probability, each point can be assigned to one of those primitives shapes. This step will follow the algorithm proposed by Schnabel et al. in \cite{schnabel-2007-efficient} and \cite{schnabel-2007-ransac}. 
\\
These primitive shapes enrich the point-cloud with additional semantic information that can be utilized to assist the users' interactions. This thesis describes two exemplary interactions that are simplified by the use of semantic information. 
Picking is the method of selecting a single point from the point-cloud by using the cursors position as input. It is challenging to hit the exact position of a projected point with the cursor. Snapping simplifies Picking by selecting the closest projected point to the cursor within a pick radius. However, this sometimes leads to the behavior that a point in the back is favored instead of a desired point in the foreground. \textit{Assisted point snapping} utilizes the geometric information of the currently focused region, such that the cursor preferably snaps to points that belong to the detected geometric shapes. 
\\
The second interaction deals with the selection of regions. The task of precisely selecting regions of interest in point-clouds can be tedious and frustrating. Using 2D-interaction metaphors only, it is challenging to select spatially neighboring points that share semantic information (e.g. \textit{points on the same plane}), as the system does not know the desired depth boundaries of the selection region. Interactions across multiple views are needed to achieve such a selection. \textit{Assisted region selection} performs a selection on the point-cloud, selecting all points that belong to a certain geometric shape, thus making it possible to interact via the detected shapes with the point-cloud, rather than the points directly. 